
\subsection{Problem Statement}

Given a deterministic system with known process dynamics, the optimal
kinodynamic motion planning problem is to find a trajectory that
begins in a given start state, terminates in a goal region, avoids
obstacles, and satisfies differential dynamics constraints. Let the
state space and control input space of the system be described by
compact sets, $X \subseteq \mathbb{R}^n$ and $U \subseteq
\mathbb{R}^m$, respectively. The system process dynamics are described
by a continuously differentiable function,
\begin{equation}
\dot{x}(t) = f(x(t),u(t)),
\label{eqn:sys_dynamics}
\end{equation}
where $x(t) \in X$ and $u(t) \in U$. Given a time horizon, $T \in
\mathbb{R}_{>0}$, a dynamically feasible trajectory is a continuous
function, $x: [0,T] \to X$, for which there exists a control function,
$u:[0,T] \to U$, that satisifies Equation~\ref{eqn:sys_dynamics}. The
optimal kinodynamic motion planning problem is to find a dynamically
feasible trajectory starts in an initial state, $x_0 \in X$, avoids
obstacles, $X_{obs} \subset X$, and terminates in a goal region,
$x_{goal} \subset X$ while minimizing a cost functional of
Equation~\ref{eqn:cost_fn}.

\begin{problem}
\label{prob:1}
(Optimal kinodynamic motion planning~\cite{Karaman.Frazzoli:CDC10})

Given a state space, $X$, obstacle region, $X_{obs}$, goal region,
$X_{goal}$, control input space, $U$, and function $f$ describing the
system process dynamics, find a control, $u : [0,T] \to U$, for some $T
\in \mathbb{R}_{>0}$ such that the corresponding dynamically feasible
trajectory, $x : [0,T] \to X$, avoids the obstacle region, reaches the
goal region, and minimizes the cost functional
\begin{equation}
J(x,u) = \int_{t=0}^T g(x(t),u(t)).
\label{eqn:cost_fn}
\end{equation}

\end{problem}


\subsection{RRT$^*$}

Optimal Rapidly-exploring random tree (RRT$^*$) \cite{karaman.frazzoli.ijrr11} is a version of the RRT algorithm~\cite{lavalle.kuffner.ijrr01} that has the \emph{asymptotic optimality} property, i.e., almost-sure convergence to an optimal solution.

The major components of the algorithm are:
\begin{itemize}
\item {\it Random sampling}:
%
The ${\tt Sample}$ procedure provides independent uniformly distributed random samples of states from the configuration space.

\item {\it Near nodes}:
%
Given a set $V$ of vertices in the tree and a state $x$, the ${\tt
Near}(V, x)$ procedure provides a set of states in $V$ that are close to
$x$. Here, closeness is measured with respect to some metric as
follows:
$$
{\tt Near}(V, x) := \left\{ x' \in V \,\,:\,\, \Vert x - x' \Vert \le \gamma
\left(\frac{\log n}{n}\right)^{1/d} \right\},
$$
where $\Vert \cdot \Vert$ is the distance using the metric, $n$ is the number of vertices in the tree, $d$ is the dimension of the space, and $\gamma$ is a constant.
 
\item {\it Steering}:
%
Given two states $x, x'$, the ${\tt Steer}(x,x')$ procedure returns a path $\sigma$ that connects $x$ and $x'$. Most implementations connect two given states with a straight path in configuration space.

\item {\it Collision checking}:
%
Given a path $\sigma$, ${\tt CollisionFree}(\sigma)$ returns true if the path lies in the obstacle-free portion of configuration space.

\end{itemize}

\begin{center}
\begin{algorithm}
\For {$i =1, \ldots, N$} { \label{line:iteration_start_orig}
$x_\mathrm{rand} \leftarrow {\tt Sample}$\; \label{line:rrtstar:sample}
$X_\mathrm{near} \leftarrow \mathtt{Near}(V ,x_\mathrm{rand})$\; \label{line:rrtstar:compute_near_orig}
$(x_\mathrm{min},\sigma_\mathrm{min}) \leftarrow {\tt ChooseParent}(X_\mathrm{near}, x_\mathrm{rand})$\; \label{line:rrtstar:call_find_min_cost_parent}
\If{${\tt CollisionFree}(\sigma)$}{ \label{line:rrtstar:check_collision}
$V \leftarrow V \cup \{ x_\mathrm{rand} \}$\;
$E \leftarrow E \cup \{ (x_\mathrm{min}, x_\mathrm{rand})$ \}\;
$(V,E) \leftarrow {\tt Rewire}(\,(V,E), X_\mathrm{near}, x_\mathrm{rand} \,)$\;
}
}
%}
\Return {$G = (V,E)$}\;
\caption{${\tt RRT}^* ((V,E), N)$}
\label{algorithm:rrtstar_orig}
\end{algorithm}
\end{center}

\vspace*{-.4in}

\begin{center}
\begin{algorithm}
$\mathrm{minCost} \leftarrow \infty$;\,
$x_\mathrm{min} \leftarrow {\tt NULL}$;
$\sigma_\mathrm{min} \leftarrow {\tt NULL}$\;
\For{$x_\mathrm{near} \in X_\mathrm{near}$}{
$\sigma \leftarrow {\tt Steer}(x_\mathrm{near}, x_\mathrm{rand})$\;
\If {${\tt Cost}(x_\mathrm{near}) + {\tt Cost}(\sigma) < \mathrm{minCost}$} {
$\mathrm{minCost} \leftarrow {\tt Cost}(x_\mathrm{near}) + {\tt Cost}(\sigma)$\;
$x_\mathrm{min} \leftarrow x_\mathrm{near}$;
$\sigma_\mathrm{min} \leftarrow \sigma$\;
}
}
\Return{$(x_\mathrm{min},\sigma_\mathrm{min})$}\;
\caption{${\tt ChooseParent}(X_\mathrm{near}, x_\mathrm{rand})$}
\label{algorithm:find_parent_orig}
\end{algorithm}
\end{center}

\vspace*{-.4in}

\begin{center}
\begin{algorithm}[h!!!]
\For{$x_\mathrm{near} \in X_\mathrm{near}$} {
$\sigma \leftarrow {\tt Steer}(x_\mathrm{rand}, x_\mathrm{near})$ \;
\If{${\tt Cost}(x_\mathrm{rand}) + {\tt Cost}(\sigma) < {\tt Cost}(x_\mathrm{near})$}{
\If {${\tt CollisionFree}(\sigma)$}{
$x_\mathrm{parent} \leftarrow {\tt Parent}(x_\mathrm{near})$\;
$E \leftarrow E \setminus \{ x_\mathrm{parent}, x_\mathrm{near}\}$\;
$E \leftarrow E \cup \{ x_\mathrm{rand}, x_\mathrm{near}\}$\;
}
}
}
\Return{$(V,E)$}\;
\caption{${\tt Rewire}( \, (V,E),X_\mathrm{near}, x_\mathrm{rand} \, )$}
\label{algorithm:rewire_orig}
\end{algorithm}
\end{center}
\vspace*{-.4in}

The algorithm proceeds as follows.
Firstly, a state is sampled, denoted as
$x_\mathrm{rand}$, from the configuration space
(Line~\ref{line:rrtstar:sample}), and near neighbors are evaluated
using the ${\tt Near}$ function
(Line~\ref{line:rrtstar:compute_near_orig}).\footnote{Although it is not
  explicitly noted here,
% in the algorithm description in order not to
 % complicate the discussion,
 we assume that the algorithm proceeds to
  the next iteration if the ${\tt Near}$ function returns an empty
  set.} Next, RRT$^*$ calls ${\tt ChooseParent}$
(Algorithm~\ref{algorithm:find_parent_orig}) to find a candidate for a
parent node to $x_\mathrm{rand}$
(Line~\ref{line:rrtstar:call_find_min_cost_parent}). ${\tt
ChooseParent}$ searches through the nodes in the set
$X_\mathrm{near}$ of near nodes and returns the node, denoted as
$x_\mathrm{min}$, that reaches $x_\mathrm{rand}$ with minimum cost
along with the path $\sigma_\mathrm{min}$ connecting
$x_\mathrm{min}$ to $x_\mathrm{rand}$. After that node
%the node within
%$X_\mathrm{near}$ that reaches $x_\mathrm{rand}$ (possibly with a path that is {\em not}
%collision free)
is found, the algorithm checks $\sigma_\mathrm{min}$ for collision
(Line~\ref{line:rrtstar:check_collision}). If the path
is obstacle-free,
%$\sigma_\mathrm{min}$ collides with an obstacle, then the algorithm
%proceeds with the next iteration. Otherwise,
the algorithm adds
$x_\mathrm{rand}$ to the tree and connects $x_\mathrm{min}$ to
$x_\mathrm{rand}$, and attempts to ``rewire'' the nodes in
$X_\mathrm{near}$ using the ${\tt Rewire}$ procedure
(Algorithm~\ref{algorithm:rewire_orig}). The ${\tt Rewire}$
procedure attempts to connect $x_\mathrm{rand}$ with each node in the
set $X_\mathrm{near}$ of near nodes. If the path that connects
$x_\mathrm{rand}$ with a near node $x_\mathrm{near}$ reaches
$x_\mathrm{near}$ with cost less than that of its current parent, then
the $x_\mathrm{near}$ is ``rewired'' to $x_\mathrm{rand}$ by
connecting $x_\mathrm{rand}$ and $x_\mathrm{near}$.

\subsection{Affine LQR}
\label{sect:affine_lqr}

Linear quadratic regulation (LQR) is an optimal control technique that
efficiently calculates an optimal policy for linear systems with
quadratic cost functions. This paper uses finite-horizon LQR that
solves the following problem: given deterministic linear process
dynamics,
\begin{equation}
\dot{x}(t) = Ax(t) + Bu(t)),
\label{eqn:linear_system}
\end{equation}
find a state and control trajectory, $x$ and $u$, that minimizes
the cost functional,
\begin{align} \nonumber
J_{LQR}(x,u) = & x(T)^T Q_F x(T) \\
& + \int_{t=0}^{t=T} x(t)^T Q x(t) + u(t)^T R u(t),
\label{eqn:lqr_cost}
\end{align}
where we have initial and final value constraints on the state
trajectory, $x(0) = x_0$ and $x(T) = x_F$. The linear quadratic
regulator solves for the optimal solution to the above problem in two
steps. First, it calculates the optimal value function by integrating
the differential Riccati equation backward in time starting at the
final time $T$ with $P(T) = Q_F$ and integrating toward $0$:
\[
-\dot{P}(t) = A^T P(t) + P(t) A - P(t) B R^{-1} B^T P(t) + Q.
\]
Second, LQR calculates the optimal action to take at time $t \leq T$
using:
\[
u(t) = -R^{-1} B^T P x.
\]

The standard LQR formulation requires process dynamics to be linear
and the goal state to be at the origin ({\em i.e.} the state cost
function to be measured with respect to the origin). However, our RRT
application requires goals in locations different than the point about
which the system is linearized. Also, we allow for affine process
dynamics. Specifically, suppose that the process dynamics are now
\begin{equation}
\dot{x}(t) = A x(t) + B u + c,
\label{eqn:affine_dynamics}
\end{equation}
and instead of minimizing Equation~\ref{eqn:lqr_cost}, suppose that
our goal is now to minimize
\begin{align} \nonumber
J_{aff}(x,u) & = (x(T)-x^*)^T Q_F (x(T)-x^*) \\
& + \int_{t=0}^{t=T}
(x(t)-x^*)^T Q (x(t)-x^*) + u(t)^T R u(t).
\label{eqn:affine_lqr_cost}
\end{align}
The difficulty is that the cost function is no longer a quadratic
form. It turns out that this problem can easily be solved by
reformulating the problem using a change of coordinates in state space
and an augmentation of the state vector. Redefine state to be
$\bar{x}(t) = x(t) - x^*$, and augment the new state vector with an
additional coordinate that will always be equal to one. The new
process dynamics are:
\[
\left(
\begin{array}{c}
\dot{\bar{x}}(t) \\
1
\end{array}
\right)
=
\left(
\begin{array}{cc}
A & c + Ax^* \\
0 & 1
\end{array}
\right)
\left(
\begin{array}{c}
\bar{x}(t) \\
1
\end{array}
\right)
+ 
\left(
\begin{array}{c}
B \\
0
\end{array}
\right)
u.
\]
It may be verified that these process dynamics are the same as those
in Equation~\ref{eqn:linear_system} except that they are now expressed
in terms of the new variable. We can now express the offset cost
functional in Equation~\ref{eqn:affine_lqr_cost} as:
\begin{align} \nonumber
J(\bar{x},u) = & 
\left(
\begin{array}{c}
\bar{x}(T) \\
1
\end{array}
\right)
\bar{Q}_F
\left(
\begin{array}{c}
\bar{x}(T) \\
1
\end{array}
\right) \\
& + \int_{t=0}^{t=T}
\left(
\begin{array}{c}
\bar{x}(t) \\
1
\end{array}
\right)
\bar{Q}
\left(
\begin{array}{c}
\bar{x}(t) \\
1
\end{array}
\right)
+ u(t)^T R u(t),
\label{eqn:affine_lqr_cost}
\end{align}
where 
$\bar{Q}_F = \left(
\begin{array}{cc}
Q_F & 0 \\
0 & 1
\end{array}
\right)$, and 
$\bar{Q} = \left(
\begin{array}{cc}
Q & 0 \\
0 & 1
\end{array}
\right)$ and we solve the affine version of the problem by applying
LQR to the augmented system in the standard way.

