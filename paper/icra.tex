%% LyX 2.0.0 created this file.  For more info, see http://www.lyx.org/.
%% Do not edit unless you really know what you are doing.
\documentclass[letterpaper, 10pt, english, conference]{IEEEtran}

\overrideIEEEmargins

\usepackage{amsfonts}
\usepackage{amsmath,amssymb}
\usepackage[tight]{subfigure}
\usepackage{cite}
\usepackage[vlined,linesnumbered,ruled]{algorithm2e}
\usepackage{color}

\ifx\pdfoutput\undefined
\usepackage{graphicx} % not running pdftex
\else
\ifx\pdfoutput\relax
\usepackage{graphicx} % not running pdftex
\else
\ifnum\pdfoutput>0
\usepackage[pdftex]{graphicx} % running pdftex with pdf output
\else
\usepackage{graphicx} % running pdftex with dvi output
\fi
\fi
\fi


%\usepackage{lmodern}
%\renewcommand{\sfdefault}{lmss}
%\renewcommand{\ttdefault}{lmtt}
\usepackage[T1]{fontenc}
\usepackage[latin9]{inputenc}
\usepackage{geometry}
\geometry{verbose}
\usepackage{fancyhdr}
\pagestyle{fancy}
\usepackage{verbatim}


%\makeatletter
%%%%%%%%%%%%%%%%%%%%%%%%%%%%%% User specified LaTeX commands.
\usepackage[T1]{fontenc}
\usepackage{pslatex}

\makeatother

\usepackage{babel}
\begin{document}

\global\long\def\at#1{}

\newcommand{\rrtstar}{RRT$^*$ }



\title{\LARGE \bf AQR-RRT$^*$: Optimal Sample-Based Motion Planning for General Dynamical Systems}

\maketitle

\begin{abstract}
We propose a method for finding probabilistically optimal control
trajectories for dynamical systems with differentiable dynamics and
twice differentiable cost functions. We first introduce a 
version of the \rrtstar algorithm that uses an LQR-based
extension to
plan for problems with affine dynamics and second order costs. 
We prove that this method almost surely
converges to the optimal plan.
We then show how to automatically derive an
extension method for problems with nonlinear dynamics and
non-quadratic costs by locally approximating both functions. This
allows us to automatically derive an extension method for 
general domains, provided their dynamics are locally differentiable  and
their cost function locally twice differentiable. This 
removes the need for
domain-specific  extension methods, which are a major
obstacle to the general applicability of RRT$^*$. 
We demonstrate
our method's application in a few standard benchmark domains. 
\end{abstract}

\section{Introduction}

intro: rrtstar is great.
offers probabilistically complete motion planning algorithm
using a sampling based method based on the popular
RRT algorithm \cite{lavalle.kuffner.ijrr01}. Yada yada.

however, we require a 

\rrtstar  \cite{karaman.frazzoli.ijrr11}

Recently, RRT, a sampling-based, probablistically complete motion
planning algorithm was extended to \rrtstar which provides probabilistic
optimality. It is straightforward to apply \rrtstar to kinematic motion
planning problems and this is typically the state-of-the-art solution.
For problems with differential constraints the application is not
so straightforward -- some of the primitives require domain-specific
design. We present a method that works well across many domains with
differential constraints.

\begin{comment}
Explain the distinction between general differential constraints and
kinodynamic constraints
\end{comment}



\section{Background}


\subsection{RRT{*}}

The RRT{*} algorithm is a sample-based motion planner to finds probabalistically
optimal solutions. It requires the following algorithmic primitives:

Steer: $ $$\left(\mbox{State}\, s_{\mbox{from}},\mbox{State}\, s_{\mbox{to}}\right)\mapsto\mbox{Action }a$
where $a$ is an action that can be applied in $s_{\mbox{from}}$
in order to get to $s_{\mbox{to}}$

Distance: $\left(\mbox{State}\, s_{\mbox{from}},\mbox{State}\, s_{\mbox{to}}\right)\mapsto\mbox{Cost }c$
where $c$ is the cost of getting from $s_{\mbox{from}}$ to $s_{\mbox{to}}$
without obstacles

CollisionFree: $\left(\mbox{State}\, s_{\mbox{from}},\mbox{State}\, s_{\mbox{to}}\right)\mapsto\mbox{StateTraj }\left\langle s_{1},\cdots,s_{T}\right\rangle $


\subsection{Modifications to Primitives}

Modification

SteerApprox: $ $$\left(x_{\mbox{from}},x_{\mbox{to}}\right)\mapsto u$
where $a$ is an action that can be applied in $x_{\mbox{from}}$
in order to get toward $s_{\mbox{to}}$

SimForward: $ $$\left(x_{\mbox{from}},u\right)\mapsto x_{\mbox{final}}$
where $u$ is an action that when applied in $x_{\mbox{from}}$ brings
the system to $x_{\mbox{final}}$

Distance: $\left(x_{\mbox{from}},x_{\mbox{to}}\right)\mapsto c$ where
$c$ is the cost of getting from $s_{\mbox{from}}$ to $s_{\mbox{to}}$
without obstacles

\begin{comment}
CollisionFree: $\left(x_{\mbox{from}},u\right)\mapsto\left\langle x_{1},\cdots,x_{T}\right\rangle ,\left\langle u_{0},\cdots,u_{T-1}\right\rangle $
where $ $$\left\langle x_{1},\cdots,x_{T}\right\rangle $ is a trajectory
of states arising from taking $u$ from $x_{\mbox{from}}$ until hitting
an obstacle or completing the action and $\left\langle u_{0},\cdots,u_{T-1}\right\rangle $
is a decomposition of $u$ that corresponds to the state trajectory. 
\end{comment}


CollisionFree: $\left(x_{\mbox{from}},u\right)\mapsto x_{\mbox{final}},u'$
where $x_{\mbox{final}}=x_{\mbox{from}}$ and $u=u'$ if the trajectory
gotten from taking $u$ in $x_{\mbox{from }}$ is entirely collision
free. If the trajectory is partially collision free, then $x_{\mbox{final}}$
is the the last state collision-free state along the trajectory before
which all states are also collision free and $u'$ is the part of
$u$ that brings $x_{\mbox{from }}$ to $x_{\mbox{final}}$.

Cost: $\left(x_{\mbox{from}},u\right)\mapsto\mbox{Cost }c$ where
$c$ is the cost of the trajectory from taking $u$ in $x_{\mbox{from}}$.

\begin{comment}
Include in code adding points along an extension?
\end{comment}


\begin{center}
\begin{algorithm}
\For {$i =1, \ldots, N$} {
    $x_\mathrm{rand} \leftarrow {\tt Sample}$\;
    ($x_\mathrm{nearest}, c_\mathrm{nearest}) \leftarrow \mathtt{Nearest}(V ,x_\mathrm{rand})$\;
    \If {$c_\mathrm{nearest} \neq \infty$} {
        $u_\text{nearest} \leftarrow {\tt SteerApprox}(x_\mathrm{nearest}, x_\mathrm{rand})$\;
        ($\sigma, x_\text{new}) \leftarrow {\tt CollisionFree}(x_\mathrm{nearest}, u_\mathrm{nearest})$\;
        \If {$\sigma \neq \emptyset$}{
            $X_\mathrm{near} \leftarrow \mathtt{Near}(V ,x_\mathrm{new})$\;
            $x_\mathrm{min},u_\mathrm{min} \leftarrow {\tt ChooseParent}(X_\mathrm{near}, x_\mathrm{new})$\;
            $x_\mathrm{new} \leftarrow {\tt SimForward}(x_\mathrm{min}, u_\mathrm{min})$\;
            $X \leftarrow X \cup \{ x_\mathrm{new} \}$\;
            $E \leftarrow E \cup \{ (x_\mathrm{min}, x_\mathrm{new})$\}\;
            $(V,E) \leftarrow {\tt Rewire}(\,(V,E), X_\mathrm{near}, x_\mathrm{new} \,)$\;
}
}
}
\Return {$G = (V,E)$}\;
\caption{${\tt ALG}^* ((V,E), N)$}

\end{algorithm}
\end{center}

\begin{center}
\begin{algorithm}
$\mathrm{minCost} \leftarrow \infty$;\,
$x_\mathrm{min} \leftarrow {\tt NULL}$;
$\sigma_\mathrm{min} \leftarrow {\tt NULL}$\;
\For{$x_\mathrm{near} \in X_\mathrm{near}$}{
    $u \leftarrow {\tt SteerApprox}(x_\mathrm{near}, x_\mathrm{new})$\;
    ($x^{\prime},u^{\prime}) \leftarrow {\tt CollisionFree}(x_\mathrm{near}, u)$\;
    \If {${\tt Cost}(x_\mathrm{near}) + {\tt Cost}(\sigma) < \mathrm{minCost} $} {
        $\mathrm{minCost} \leftarrow {\tt Cost}(x_\mathrm{near}) + {\tt Cost}(\sigma)$\;
        $x_\mathrm{min} \leftarrow x_\mathrm{near}$;
        $\sigma_\mathrm{min} \leftarrow \sigma$\;
    }
}
\Return{$(x_\mathrm{min},\sigma_\mathrm{min})$}\;
\caption{${\tt ChooseParent}(X_\mathrm{near}, x_\mathrm{new})$}

\end{algorithm}
\end{center}

\begin{center}
\begin{algorithm}
\For{$x_\mathrm{near} \in X_\mathrm{near}$} {
    $\sigma \leftarrow {\tt LQRSteer}(x_\mathrm{new}, x_\mathrm{near})$ \;
    \If{${\tt Cost}(x_\mathrm{new}) + {\tt Cost}(\sigma) < {\tt Cost}(x_\mathrm{near})$}{
        \If {${\tt CollisionFree}(\sigma)$}{   
            $x_\mathrm{parent} \leftarrow {\tt Parent}(x_\mathrm{near})$\;
            $E \leftarrow E \setminus \{ x_\mathrm{parent}, x_\mathrm{near}\}$\;
            $E \leftarrow E \cup \{ x_\mathrm{new}, x_\mathrm{near}\}$\;
}
    }
}
\Return{$(V,E)$}\;
\caption{${\tt Rewire}( \, (V,E), X_\mathrm{near}, x_\mathrm{new} \, )$}
\end{algorithm}
\end{center}

The changes to the primitives do not logically change the \rrtstar algorithm,
but instead make it easier to plug-in the contributed components.


\subsection{Math Tools}

Linear Quadratic Regulator (LQR) is an optimal control technique which
determines the policy (mapping from state to action) for a linear
dynamical system which minimizes a quadratic functional on the state
and action.

LQR is used to solve the following problem: 

Find $\left\langle u_{0},\cdots u_{T-1}\right\rangle $ such that 

\[
x_{k+1}=Ax_{k}+Bu_{k}
\]


minimizing 
\[
J\left(x,u\right)=\sum_{k=0}^{T-1}\left(x_{k}^{T}Qx_{k}+u_{k}^{T}Ru_{k}\right)+x_{T}^{T}Q_{T}x_{T}
\]


given $A$, $B$, $Q$, $R$, $x_{0}$, $x_{T}$, and $T$.

We can enforce that LQR policies obey the final-value constraint on
$x_{T}$ by placing very high cost (practically infinite) on the $Q_{T}$
fed to the finite-horizon LQR solver. The textbook LQR problem statement
requires that the quadratic bowl $x^{T}Qx$ be centered at $x=0$.
A shift of coordinates can change the center of all bowls, which is
not what we want.

We can enrich the class of LQR problems to have arbitrary second-order
penalty (along with first-order dynamics, which will be essential
later) by considering the follwing transformation:

\begin{align*}
\underbrace{\left[\begin{matrix}x_{k+1}\\
1
\end{matrix}\right]}_{\hat{x}_{k+1}} & =\underbrace{\left(\begin{matrix}A & c-R^{-1}r\\
0 & 1
\end{matrix}\right)}_{\hat{A}}\underbrace{\left[\begin{matrix}x_{k}\\
1
\end{matrix}\right]}_{\hat{x}_{k}}+\underbrace{\left(\begin{matrix}B\\
0
\end{matrix}\right)}_{\hat{B}}\hat{u}_{k}\\
J\left(x,u\right) & =\sum_{k=0}^{T-1}\hat{x}_{k}^{T}\hat{Q}\hat{x}_{k}+\hat{u}_{k}^{T}R\hat{u}_{k}+\hat{x}_{T}^{T}\hat{Q}_{T}\hat{x}_{T}
\end{align*}


with $\hat{u}_{k}=u_{k}+R^{-1}r$ and $\hat{Q}=\left(\begin{matrix}Q & q\\
q^{T} & d
\end{matrix}\right)$ . 

Here we augment the state vector with an additional dimension whose
value is constrained by the dynamics to remain constant. As long as
this additional dimension is unity in $x_{0}$ , then it will be unity
along all trajectories.

In addition to allowing first-order dynamics (increasing the class
of systems from linear to affine), this transformation also allows
an arbitrary second-order function on $x$ and $u$. For example,
say we want the following bowl:

\[
\left(x-x_{0}\right)^{T}Q\left(x-x_{0}\right)=x^{T}Qx+-2x_{0}^{T}Qx+x_{0}^{T}Qx_{0}
\]


Then choose $\hat{Q}=\left(\begin{array}{cc}
Q & -Qx_{0}\\
-x_{0}^{T}Q & x_{0}^{T}Qx_{0}
\end{array}\right)$.


\section{Application to a constrained Linear System with Quadratic Cost }

Necessary since in general, not possible or practical for steer function
to be exact. 

Logically equivalent to common formulation of \rrtstar, but easier to
plug in the primitives.

Algorithm detailing the different structure here 

For this class of systems, the LQR method is the standard way to find
optimal solution. LQR, though, cannot handle constaints on the state
nor on actuation.

We can use LQR in the distance metric calculation.


\subsection{Augmenting State with Time}

Augmenting time in the \rrtstar state space allows us to set explicit
time goals and allows us to apply the LQR heuristic. Without this,
it's not clear what the cost between two states should be. Penalizing
time may not be the metric we want to optimize for. Choosing the lowest
cost over all possible time horizons does not guarantee completeness.
These are crucial points.

Many scenarios include time in the state anyway. For example: time-varying
dynamics or obstacles and constraints. 

Let $\left\langle x,k\right\rangle \in\mathbb{R}^{n}\times\mathbb{R}_{0+}$
be the space of the RRT state.


\subsubsection{Primitives}

$Steer(s_{1},s_{2})$

$T=k\left(s_{2}\right)-k\left(s_{1}\right)$

if $T\leq0$, return NullAction

follow LQR gain matrices around $x\left(s_{1}\right)$ with goal $x\left(s_{2}\right)$
, time horizon $T$

$Distance(s_{1},s_{2})$

use LQR cost-to-go


\subsubsection{Optimality Proof Sketch}

LQR solves a relaxed version of the problem -- no obstacle and no
actuation constraints. This is directly analogous to a Euclidean distance
metric being a relaxed version of the shortest path in the kinematic
case.


\section{Application to General Dynamical Systems}

Consider a discrete-time dynamical system in the form 
\begin{equation}
x_{k+1}=f\left(x_{k},u_{k}\right)\label{eq:dynamics}
\end{equation}


with additive cost function
\begin{equation}
J\left(\mathbf{u},\mathbf{x}\right)=\sum_{k=0}^{T}g\left(x_{k},u_{k}\right)\label{eq:cost_functional}
\end{equation}


and starting point $x_{0}$. The state vector, $x$, is $n$-dimensional
and the control vector, $u$, is $m$-dimensional. We aim to find
a sequence $ $$\mathbf{u}=\left\{ u_{0},\cdots,u_{T}\right\} $ which
induces a trajectory $\mathbf{x}=\left\{ x_{1},\cdots,x_{T}\right\} $
satisfying the dynamics (\ref{eq:dynamics}) such that $ $$C$ is
minimized according to (\ref{eq:cost_functional}).

The real cost of moving from point $x'$ to $x''$ is
\begin{multline*}
C\left(x,x'\right)=\min_{\mathbf{u}}J\left(\mathbf{u},\mathbf{x}\right)\\
\mbox{subject to (\ensuremath{\ref{eq:dynamics}}), \ensuremath{x_{0}=x}, \ensuremath{x_{T}=x'}}
\end{multline*}


Note that the minimization happens over control sequences $\mathbf{u}$
of a fixed time lengths, according to 

We approximate $C\left(x',x''\right)$ by taking a first-order approximation
of the dynamics and a second-order approximation of the cost and applying
LQR control. In general, the approximated dynamics and cost are of
the following form

\begin{align}
x_{k+1} & \approx Ax_{k}+Bu_{k}+c\label{eq:dynamics_approx}\\
J\left(\mathbf{u},\mathbf{x}\right) & \approx\sum_{k=0}^{T}x_{k}^{T}Qx_{k}+u_{k}^{T}Ru_{k}+2q^{T}x_{k}+2r^{T}u_{k}+d\label{eq:cost_approx}
\end{align}


$A$ and $Q$ are $n\times n$, $B$ is $m\times n$, $R$ is $n\times n$.
$c$ and $q$ are $n\times1$, $r$ is $m\times1$ and $d$ is a scalar.

\begin{align*}
A & =\left.\frac{\partial f}{\partial x}\right|_{x^{*},u^{*}}\\
B & =\left.\frac{\partial f}{\partial u}\right|_{x^{*},u^{*}}\\
c & =-Ax^{*}-Bu^{*}+f\left(x^{*},u^{*}\right)
\end{align*}


$x^{*}$, $u^{*}$ is the point about which the linearization is performed.
Typically $u^{*}$ is taken to be $\mathbf{0}$ and $x^{*}=x'$

Equations \ref{eq:dynamics_approx} and \ref{eq:cost_approx} are
the truncated Taylor expansions of $f$ and $g$. The dynamics $f$
must be once-differentiable and addition cost $g$ must be twice-differentiable.


\subsection{Reduction to the Previous Problem}

It is possible to transform the problem specified with \ref{eq:dynamics_approx}
and \ref{eq:cost_approx} into LQR form (where there is only an $A$,
$B$, $Q$, $R$ matrix) using the following:

\begin{align*}
\underbrace{\left[\begin{matrix}x_{k+1}\\
1
\end{matrix}\right]}_{\hat{x}_{k+1}} & =\underbrace{\left(\begin{matrix}A & c-R^{-1}r\\
0 & 1
\end{matrix}\right)}_{\hat{A}}\underbrace{\left[\begin{matrix}x_{k}\\
1
\end{matrix}\right]}_{\hat{x}_{k}}+\underbrace{\left(\begin{matrix}B\\
0
\end{matrix}\right)}_{\hat{B}}\hat{u}_{k}\\
C\left(\mathbf{u},\mathbf{x}\right) & =\sum_{k=0}^{T}\hat{x}_{k}^{T}\hat{Q}\hat{x}_{k}+\hat{u}_{k}^{T}R\hat{u}_{k}
\end{align*}


with $\hat{u}_{k}=u_{k}+R^{-1}r$ and $\hat{Q}=\left(\begin{matrix}Q & q\\
q^{T} & d
\end{matrix}\right)$ . 

\begin{comment}
$\left(\begin{matrix}x\\
1
\end{matrix}\right)^{T}\left(\begin{matrix}Q & q\\
q^{T} & \eta_{1}
\end{matrix}\right)\left(\begin{matrix}x\\
1
\end{matrix}\right)=\left(\begin{matrix}x\\
1
\end{matrix}\right)^{T}\left(\begin{matrix}Qx+q\\
q^{T}x+\eta_{1}
\end{matrix}\right)=x^{T}Qx+x^{T}q+q^{T}x+d$
\end{comment}


The $\hat{A}$, $\hat{B}$, $\hat{Q}$, and $R$ matrices specify
a linear dynamical system with quadratic costs to which an optimal
solution can be found with LQR. 


\subsection{Nuances and Subtleties}


\subsubsection{Non-exact steering}

rewiring and propagating dynamics


\subsubsection{Uncontrollable Dynamics}

The linearized system may be uncontrollable -- the $A$ and $B$ matrices
are such that it's not possible to control all the modes of the system.
This is the case, for example, for a cart with two inverted pendulums
of the same length linearized about the upward-pointing fixed point.
The control input to the system affects both linearized pendulums
in the same way, so it's not possible to independently stabilize them.
For the infinite-horizon LQR control problem, there is no solution.
For the finite-horizon problem, there is a solution, though it might
not be possible to go to any arbitrary location. If the system linearized
at $x'$ cannot reach $x''$, then $C\left(x',x''\right)$ needs to
be defined in another way.

is Therefore using the LQR cost metric cannot approximate the cost 

\begin{comment}
Typical RTT{*} the cost metric is an underestimate of the true cost,
since the metric does not take into consideration obstacles. The true
cost, for example, might be infinite if there is no feasible path,
but the Euclidian metric will always be finite.

If the system is uncontrollable as is linearized by a single point,
then the LQR cost will be infinite while the true cost is not.
\end{comment}



\subsubsection{Indefinite Cost}

\begin{comment}
What if $\hat{Q}$ is indefinite? 

Options:

1. Use sequential QP techniques like shifting eigenvalues to make
$\hat{Q}$ definite. 

http://www.cs.berkeley.edu/\textasciitilde{}pabbeel/cs287-fa11/slides/NonlinearOptimizationForOptimalControl.pdf
(Page 11 bottom slide)

2. LQR with indefnite weighting matrices

Chapter 9 of:

http://epubs.siam.org.libproxy.mit.edu/doi/book/10.1137/1.9781611970760

(same book) 

http://books.google.com/books?id=bD\_83idGZ2cC\&lpg=PA211\&ots=q3U7u4rmNc\&dq=indefinite\%20LQR\&pg=PA211\#v=onepage\&q\&f=false

http://www.tandfonline.com/doi/abs/10.1080/00207178408933184\#preview
\end{comment}



\subsubsection{Actuation Constraints}

The LQR framework does not permit actuation constraints.

\begin{comment}
Say the LQR solution returns a control action $u\not\in\mathcal{U}$
(the set of permitted actions). Simply choosing a $u'=\alpha u$ such
that $u'\in\mathcal{U}$ while minimizing $\alpha^{2}$ (so that $u'$
is close to $u$) intuitively will not explore the space as desired
-- the state won't move along the same direction in general.

To clarify:

if the state of the system is $x_{k}$, then the next state is $x_{k+1}=Ax_{k}+Bu$,
which presumably moves the system toward $x_{rand}$. The state $x_{k+1}'=Ax_{k}+Bu'$
does not, in general, move the system directly toward $x_{rand}$.
\end{comment}



\subsubsection{Asymmetric Cost}

\begin{comment}
http://math.stackexchange.com/a/23397/2256
\end{comment}



\section{Related Work}

vdB, Glassman, Perez


\section{Results}


\subsection{Linear Domain}

spaceship no orientation


\subsection{Non-linear Domain}

spaceship orientation


\subsection{Results}

Quick mention of performance (or not). Picture of tree, cost over
iteration


\section{Discussion and Conclusion and Future Work}

Code available. Spatial Data structure future


\section{Acknowledgments}


\section{References}
\bibliographystyle{IEEEtran}
\bibliography{references}
\end{document}
